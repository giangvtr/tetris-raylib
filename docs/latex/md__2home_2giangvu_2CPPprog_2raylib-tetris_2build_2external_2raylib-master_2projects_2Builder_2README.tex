\chapter{Builder project template}
\hypertarget{md__2home_2giangvu_2CPPprog_2raylib-tetris_2build_2external_2raylib-master_2projects_2Builder_2README}{}\label{md__2home_2giangvu_2CPPprog_2raylib-tetris_2build_2external_2raylib-master_2projects_2Builder_2README}\index{Builder project template@{Builder project template}}
\label{md__2home_2giangvu_2CPPprog_2raylib-tetris_2build_2external_2raylib-master_2projects_2Builder_2README_autotoc_md71}%
\Hypertarget{md__2home_2giangvu_2CPPprog_2raylib-tetris_2build_2external_2raylib-master_2projects_2Builder_2README_autotoc_md71}%
 This is a project template to be used with \href{https://raw.githubusercontent.com/jubalh/raymario/master/meson.build}{\texttt{ GNOME Builder}}. It uses the \href{https://raw.githubusercontent.com/jubalh/raymario/master/meson.build}{\texttt{ meson}} build system.

Project can be compiled via the command line using\+: 
\begin{DoxyCode}{0}
\DoxyCodeLine{meson\ build}
\DoxyCodeLine{cd\ build}
\DoxyCodeLine{ninja}
\DoxyCodeLine{ninja\ install}

\end{DoxyCode}


Or it can be opened with Building simply clicking on the {\ttfamily meson.\+build} file. Alternatively, open Builder first and click on the {\ttfamily open} button at the top-\/left.

There are comments to the {\ttfamily meson.\+build} file to note the values that should be edited. For a full overview of options please check the \href{http://mesonbuild.com/Manual.html}{\texttt{ meson manual}}.

In the provided file, it\textquotesingle{}s assumed that the build file is located at the root folder of the project, and that all the sources are in a {\ttfamily src} subfolder.

Check out the {\ttfamily examples} directory for a simple example on how to use this template.

You can also look at \href{https://github.com/jubalh/raymario}{\texttt{ raymario}} for a slightly more complex example which also installs resource files. 