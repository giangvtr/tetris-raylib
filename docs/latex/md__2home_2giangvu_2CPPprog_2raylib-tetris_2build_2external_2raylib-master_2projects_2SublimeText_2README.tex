\chapter{Sublime Text 3 project template}
\hypertarget{md__2home_2giangvu_2CPPprog_2raylib-tetris_2build_2external_2raylib-master_2projects_2SublimeText_2README}{}\label{md__2home_2giangvu_2CPPprog_2raylib-tetris_2build_2external_2raylib-master_2projects_2SublimeText_2README}\index{Sublime Text 3 project template@{Sublime Text 3 project template}}
\label{md__2home_2giangvu_2CPPprog_2raylib-tetris_2build_2external_2raylib-master_2projects_2SublimeText_2README_autotoc_md86}%
\Hypertarget{md__2home_2giangvu_2CPPprog_2raylib-tetris_2build_2external_2raylib-master_2projects_2SublimeText_2README_autotoc_md86}%
 This is a project template to be used with \href{https://www.sublimetext.com/}{\texttt{ Sublime Text 3}}.

Simply click on the {\ttfamily raylib.\+sublime-\/project} file to open it with Sublime Text. Alternatively you can open Sublime Text first and click on the {\ttfamily Project -\/\texorpdfstring{$>$}{>} Open Project}.

It comes with raylib.\+sublime-\/build. This file needs to be copied into the sublime packages folder for the user. On windows it can be found in {\ttfamily App\+Data\%\textbackslash{}Sublime Text 3\textbackslash{}Packages\textbackslash{}User}.

Once done open the project and select {\ttfamily Tools -\/\texorpdfstring{$>$}{>} Build System -\/\texorpdfstring{$>$}{>} raylib}. To run press Ctrl + Shift + B and select which build you want to run.

For a full overview of build systems please check the \href{https://www.sublimetext.com/docs/3/build_systems.html}{\texttt{ build systems guide}}. 